	Long-term trends over 14 years show that academics are writing more papers, with more co-authors, and those papers are getting longer. 
	However, the increased quantity does not reflect quality: we find evidence that the level of intellectual contribution needed to warrant co-authorship on a paper has dropped in astronomy, \todo{and elsewhere?}. 
	We find that more leading and co-author papers in the first few years of an academic's career increases their chances to stay in academia, affirming the `\emph{publish or perish}' mantra, but only to a point: after an early career researcher has published enough papers, their chances of continuing in academia remain flat. At that point, they should stop writing papers and do something else.


Academics are writing more papers, and with more co-authors (Figure~\ref{fig:pre-prints-segmented-by-author-count}), with a steady decline of single author papers in every field. All papers are getting longer, too: the mean abstract length is increasing, and even the shortest abstracts each year are getting longer (Figure~\ref{fig:pre-prints-abstract-length}).




\section*{Longevity and authorship valuation}

%	\item Is this increased quantity of publications, and length of publications, a measure of increased quality of research?

We find that the number of authors on a paper has tended to increase in most fields between 2007 and 2020 (Figure~\ref{fig:number-of-authors-with-time}). These trends have been explained by the increasing complexity of research projects that requires expertise from different areas [REF]. If the increasing collaboration size is entirely due to research expertise, then we would expect that the expanding network size would mostly be reflected in established researchers that are experts in their field. Instead, we find that the minimum amount of work necessary to warrant co-authorship has decreased in some fields, which \todo{mostly} accounts for the increase in collaboration sizes with time.

Consider a motivated PhD supervisor with good intentions. If the supervisor encourages their student to contribute to a paper where the supervisor is already involved (but the supervisor is not leading), then the supervisor will be helping to improve the student's visibility (and perceived productivity) in an already competitive landscape. This is a common scenario in academia: an established researcher will assign some task to a PhD student, and the outcome of that task will contribute to a paper in preparation. As a consequence, both the student and supervisor are rightfully listed as co-authors instead of just the supervisor. If the valuation of `intellectual contribution necessary to warrant publication' remained stable with time, then we should expect that the number of co-author publications that overlap with the PhD supervisor(s) will also remain stable with time.

We identified two cohorts of early career researchers (ECRs) in astronomy that are separated by their career start date by five years (see Methods). The distribution of first-author (leading) publications between these cohorts is indistinguishable (\todo{some metrics}, Figure~\ref{fig:longevity}). The relative number of researchers with \todo{up to 5} co-author publications is also the same between both cohorts, but the more recent cohort has a long-tail of ECRs with \todo{X or more} co-author publications. These researchers would appear to be more productive, and to have a more extensive network of collaborators. When we separate the co-author publications to those that do, or do not, include the inferred PhD supervisor(s), it is clear that the increased number of co-author publications is not due to network expansion: those ECRs are publishing with their PhD supervisors and not independently expanding their network. While it is impossible to reliably identify a causal link here, it would appear that those ECR students with increased co-author publications, that overlap heavily with their supervisors, is perhaps due to the supervisors good intention of helping the student expand their network. While individual intentions are good, the net behavioural effect of this as a research field is that the intellectual merit necessary for publication can decrease on average. 

\todo{Check whether the increased collaboration size is driven by experts being co-authors, or ECRs as being co-authors}


The data suggest that those extra publications has little or no impact on an ECR's likelihood of remaining in academia. We identified astronomy ECRs that started between 2010-2012 and separated them into two cohorts: those that continued to publish in astronomy, and those that did not. For every first-author paper written by an ECR in the first \todo{X} years, their likelihood of continuing in academia increased. However, this effect flattens out at \todo{$N \approx 6$}: an ECR with 6 first author publications is just as likely to continue in academia as an ECR with 10 first author publications. A similar effect is seen in co-author publications, although the rate is lower. These two effects are related: an ECR with 10 first-author papers and no co-author papers may look as employable (or un-employable) as an ECR with 10 co-author papers and no first-author papers, but for different reasons. Matching these cohorts with citation, network, and institution metrics would likely reveal a more predictive model for whether an individual ECR will remain in academia. But as a cohort, the probability of continuing in astronomy research flattens out at about 50\%: for astronomy ECRs is the top echelon of productivity, half of them will still leave academia. 




\section*{Authorship valuation analysis}


\section*{Longevity analysis}


