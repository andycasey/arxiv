
@article{Ginsparg:2011,
  doi = {10.1038/476145a},
  url = {https://doi.org/10.1038/476145a},
  year = {2011},
  month = aug,
  publisher = {Springer Science and Business Media {LLC}},
  volume = {476},
  number = {7359},
  pages = {145--147},
  author = {Paul Ginsparg},
  title = {{ArXiv} at 20},
  journal = {Nature}
}

@article{Lariviere:2014,
   title={arXiv E-prints and the journal of record: An analysis of roles and relationships},
   volume={65},
   ISSN={2330-1635},
   url={http://dx.doi.org/10.1002/asi.23044},
   DOI={10.1002/asi.23044},
   number={6},
   journal={Journal of the Association for Information Science and Technology},
   publisher={Wiley},
   author={Larivi�re, Vincent and Sugimoto, Cassidy R. and Macaluso, Benoit and Milojevi?, Sta�a and Cronin, Blaise and Thelwall, Mike},
   year={2014},
   month={Jan},
   pages={1157?1169}
}

@misc{clement2019use,
      title={On the Use of ArXiv as a Dataset}, 
      author={Colin B. Clement and Matthew Bierbaum and Kevin P. O'Keeffe and Alexander A. Alemi},
      year={2019},
      eprint={1905.00075},
      archivePrefix={arXiv},
      primaryClass={cs.IR}
}
@article{IbnMohammed:2021,
  doi = {10.1016/j.resconrec.2020.105169},
  url = {https://doi.org/10.1016/j.resconrec.2020.105169},
  year = {2021},
  month = jan,
  publisher = {Elsevier {BV}},
  volume = {164},
  pages = {105169},
  author = {T. Ibn-Mohammed and K.B. Mustapha and J. Godsell and Z. Adamu and K.A. Babatunde and D.D. Akintade and A. Acquaye and H. Fujii and M.M. Ndiaye and F.A. Yamoah and S.C.L. Koh},
  title = {A critical analysis of the impacts of {COVID}-19 on the global economy and ecosystems and opportunities for circular economy strategies},
  journal = {Resources,  Conservation and Recycling}
}

@misc{LatticeConferenceWebsite, 
	title={The 38th International Symposium on Lattice Field Theory (Lattice 2020)}, 
	url={https://indico.hiskp.uni-bonn.de/event/1/}, 
	journal={Helmholtz-Institut f�r Strahlen- und Kernphysik Indico Service (Indico)}
} 
@book{rasmussen:williams:2006,
  added-at = {2009-03-05T08:49:50.000+0100},
  author = {Rasmussen, C. E. and Williams, C. K. I.},
  biburl = {https://www.bibsonomy.org/bibtex/26771eaebbee7d852934f29aa33dea971/bcao},
  interhash = {72c030472023000e0bdeeb06081c3764},
  intrahash = {6771eaebbee7d852934f29aa33dea971},
  keywords = {},
  publisher = {MIT Press},
  timestamp = {2009-03-05T08:49:50.000+0100},
  title = {Gaussian Processes for Machine Learning},
  year = 2006
}


@article{hodlr,
    author = {{Ambikasaran}, S. and {Foreman-Mackey}, D. and
              {Greengard}, L. and {Hogg}, D.~W. and {O'Neil}, M.},
     title = "{Fast Direct Methods for Gaussian Processes}",
      year = 2014,
     month = mar,
       url = http://arxiv.org/abs/1403.6015
}
 

@article{Nicola:2020,
  doi = {10.1016/j.ijsu.2020.04.018},
  url = {https://doi.org/10.1016/j.ijsu.2020.04.018},
  year = {2020},
  month = jun,
  publisher = {Elsevier {BV}},
  volume = {78},
  pages = {185--193},
  author = {Maria Nicola and Zaid Alsafi and Catrin Sohrabi and Ahmed Kerwan and Ahmed Al-Jabir and Christos Iosifidis and Maliha Agha and Riaz Agha},
  title = {The socio-economic implications of the coronavirus pandemic ({COVID}-19): A review},
  journal = {International Journal of Surgery}
}

@article{Chu:2020,
  doi = {10.1093/jtm/taaa192},
  url = {https://doi.org/10.1093/jtm/taaa192},
  year = {2020},
  month = oct,
  publisher = {Oxford University Press ({OUP})},
  volume = {27},
  number = {7},
  author = {Isaac Yen-Hao Chu and Prima Alam and Heidi J Larson and Leesa Lin},
  title = {Social consequences of mass quarantine during epidemics: a systematic review with implications for the {COVID}-19 response},
  journal = {Journal of Travel Medicine}
}